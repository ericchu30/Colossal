
\section{Discussion and Related Work}
\label{sec:sec4}

Efficient processing of big data has given rise to multi-tenant, big data clusters where multiple applications run on and share the same resources and data. Such multi-tenancy poses significant challenges for human operators of the cluster to tune the cluster to meet performance requirements that often conflict each other. The research and open-source communities 
are doing a lot of work in this area. The main threads of 
work include: (a) building cluster operating systems 
like Mesos \cite{mesos} and YARN \cite{yarn}
that have the resource provisioning and 
isolation mechanisms to support multi-tenancy; (b) 
algorithms and policies to allocate resources in ways that are 
efficient as well as fair to all tenants \cite{conf/cidr/NarasayyaDSCC13,ghodsi11,isard09}; and (c) ability to support multiple tenants
on as few software and hardware resources as possible 
(e.g., \cite{DBLP:journals/pvldb/DasNLS13}).

In this paper, we observed that, in practice, the responsibility 
of managing multi-tenancy falls largely on the operations team that 
manages a  multi-tenant cluster. 
We used operational experiences from a large, multi-tenant 
cluster to discuss the challenges that arise. Specifically, 
we observed that cluster operators have 
few effective tools to make data-driven decisions
while managing multi-tenancy. We described 
Colossal, a platform that we have developed to address this gap. 

Space contraints precluded us from presenting results from an experimental
evaluation. Overall, the primary advantages that Colossal 
offers are: 
\begin{itemize}

\item{\em Efficiency}: Colossal can predict, with acceptable 
levels of accuracy, the performance metrics and schedule
of one week's workload at Rocket Fuel 
(approximately, 55,000+ MapReduce jobs and 36 million attempts) in
under four minutes on a commodity machine.

\item{\em Flexibility}: Colossal allows for easy abstraction, modification, 
and prediction of the workload over any specified time period lengths.

\item{\em Modularity}: The components that generate the 
workload as well as predicted performance metrics
are pluggable. Thus, Colossal can be customized based on the 
characteristics and needs of different multi-tenant environments. 

\end{itemize}


